\chapter{Meetings}
\label{meeting_chapter}

\begin{table}[H]
    \centering
    \begin{tabular}{| c | c | c | p{0.36\linewidth} | c |}
        \hline 
        Nr & Phase & Date & Description & Duration [min]\\
        \hline \hline
        1 & Elaboration & 06.10.2022 & Coordinate the project, documentation \- and ideas & 90\\
        \hline
        2 & Elaboration & 13.10.2022 & Present the Problemdomain with Learning Concepts and define the project plan & 60 \\
        \hline
        3 & Elaboration & 20.10.2022 & Lab Concept Drafts, GANTT Diagram & 80  \\
        \hline
        4 & Construction & 27.10.2022 & Think about the exploitation aspect and add it to mindmap; POC for Lab 2 and 3 and started testing & --\\
        \hline
        5 & Construction & 03.11.2022 & ---- & --\\
        \hline
        6 & Construction & 10.11.2022 & ---- & --\\
        \hline
    \end{tabular}
    \caption{Meetings held with advisor}
    \label{meetings_hold_table}
\end{table}
\section{Elaboration}
\subsection*{Meeting 1: 06. Okt.}
\textbf{Deliverables:} Draft the Problemdomains together with Learningconcepts and the Projectplan. \\
\textbf{Discussed Topics:} 
\begin{itemize}
    \item Documentation
    \item Goal of the Project: Hacking Lab Integration for cyber security classes.
    \begin{itemize}
        \item Teacher needs required knowhow
        \item Labs built for a student attending the third year
    \end{itemize}
    \item Brainstorming for Problemdomains
    \begin{itemize}
        \item Learningconcepts need to be defined for teacher and for Lab construction
        \item RE Tools
        \item Required Knowhow for a Reverse Engineer
    \end{itemize}
    \item Mini Projectplan Brainstorming:
    \begin{itemize}
        \item Analysis of Problemdomains
        \item Setting up Learningconcepts
        \item What is important?
        \item Build Lab Concepts
        \item Implement with Tests
    \end{itemize}
\end{itemize}
\textbf{Decisions:} 
\begin{itemize}
    \item Idea of the project: Hacking Lab Integration (Modulintegration)
\end{itemize}
\textbf{Duration:} 1 hour 30 min

\subsection*{Meeting 2: 13. Okt.}
\textbf{Deliverables:} Build some Lab Concepts (deliverables and overall subject), GANTT Diagramm \\
\textbf{Discussed Topics:} 
\begin{itemize}
    \item Problemdomain Mindmap
    \item Learning Concepts
    \item What a Hacking-Lab course needs
    \begin{itemize}
        \item Deliverables defined
        \item Show which skills are trained (Mindmap / MITRA style table)
        \item Exercises created as Markdown for ease of use 
        \item Get inspiration from already established courses
    \end{itemize}
\end{itemize}
\textbf{Decisions:} 
\begin{itemize}
    \item Add passive and active labs for networks.
    \item Write protocol of sprint meetings for own use but not needed in documentation.
    \item Create Hacking-Lab content on markdown files
\end{itemize} 
\textbf{Duration:} 1 hour

\subsection*{Meeting 3: 20. Okt.}
\textbf{Deliverables:} Students: Think about the exploitation aspect and add it to mind\-map; POC for Lab 2 and 3 and started testing; Update GANTT chart. \\ 
Advisor: Sample Courses for RE; Unlock ECSC Hacking-Labs \\
\textbf{Discussed Topics:} 
\begin{itemize}
    \item Online GANTT chart instead of Excel chart
    \item Refresher and Lab 1 (Introduction \#1)
    \begin{itemize}
        \item Create at the end to know what is necessary for the labs
        \item Use binaries from the already created labs
    \end{itemize}
    \item Server part is important (analyzing and exploiting remote); socat discussed
    \item Generally try to add exploits at the end
    \item RE is used to find weaknesses and exploit them
    \item Markdown for the Hacking-Lab consists of: Main page / Step pages and a solution page
\end{itemize}
\textbf{Decisions:}
\begin{itemize}
    \item Timetracking: Only if the used tool (clockify) allows it, add estimated time
    \item Plan Labs while creating others to maximize the efficency
    \item Start construction phase now (21.10.2022)
    \item Add exploits to the labs (remote server using docker)
    \item Start construction with lab 2 and 3 to ensure the refresher and lab 1 are done correctly
\end{itemize}
\textbf{Duration:} 1 hour 20min

\section{Construction}
\subsection*{Meeting 3: 27. Okt.}
\textbf{Deliverables:} \\
\textbf{Discussed Topics:} \\
\textbf{Decisions:} \\
\textbf{Duration:}

\section{Transition}
