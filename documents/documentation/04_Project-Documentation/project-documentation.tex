\chapter{Project Documentation}
% --------------------------------------%
\section{Project Plan}
\subsection{Project Overview}
The goal of this project is to create and organize a lab, which shows and explains future students of the Ostschweizer Fachhochschule (OST) how reverse engineering is performed and which tactics are used to get information out of a program. To accomplish this task, the lab will have several exercises organized in the different domains. These exercises will be accessible through the Hacking-Lab hosted on the OST server. 

\subsubsection*{Hand-In}
The finished Report will be handed in according to the rules set by the "Studiengangsleitung Informatik" and the supervisor:
\begin{itemize}
    \item The PDF version will be sent to the advisor and to the OST archive.
    \item The printed version will be handed in to the supervisor for reading and grading.
\end{itemize}

\subsection{Management}
\subsubsection*{Time Management}
The project started on the first week of the semester (KW 38) and ends in week 51 giving us around 14 weeks to be done with the Hand-In. \\
Since the module has a total ECTS of 8 each of the students has to work around 240h during the semester which can be seen in table \ref{time_ects} together with the total planned time investment. This means, that per week each student should work around 17.1 hours.

\begin{table}
    \centering
    \begin{tabular}{||c c c c||} 
        \hline
        Name & ECTS & Time spent per Week [h] & Total Time spent [h]\\ [0.5ex] 
        \hline\hline
        Gianluca Nenz & 8 & 17.1 & 240 \\ 
        \hline
        Ronny Mueller & 8 & 17.1 & 240 \\
        \hline
        Thomas Kleb & 8 & 17.1 & 240 \\ 
        \hline
        \textbf{Total} & 32 & 52.3 & 720 \\[1ex] 
        \hline
    \end{tabular}
    \caption{Time Investments}
    \label{time_ects}
\end{table}

\subsubsection*{Planning and Project Management}
In the past modules Software Engineering Practices 1 and 2 (SEP 1 + 2) we were introduced to different ways to plan and organize a project. The main tools we learned, RUP (Rational Unified Process) and Scrum, are mainly used in software development but can be adapted to other projects aswell. They both use different aspects of time management and organisation which is why we intend to apply them to our project. \\
We use RUP to section our project into Inception, where we get a first insight into the project and how we want it to resolve; Elaboration, to plan our project, define the workload-distribution and setting up first concepts of the finished labs; The construction phase is mainly used to plan, build and test the labs while the last phase, the transition phase, is used as buffer and to finish our product. \\ 
To make sure everything works as planned we use Scrum with its Sprints to setup Milestones and Tasks which help structurize the development. 

\subsection{Organisation}

\subsubsection*{Participants}
The "Studienarbeit"-Team consists of three students: Gianluca Nenz, Ronny Muel\-ler and Thomas Kleb. Work on the project and documentation will be evenly distributed between these three participants. Bigger decisions are made as a team in either the meetings with or without the advisor (the advisor will be notified on any change made).

\subsubsection*{Advisor}
The teams advisor for the "Studienarbeit" is Ivan Buetler who is teaching cyber security modules at the OST.

\subsubsection*{Division of Labor}
The project has multiple facets that need to be taken care of. This is why the team has decided to distributed the work load between the three. This doesn't mean that the work is done by only the chosen student but rather that he is the one responsible that it works as planned.
\begin{table}[H]
    \begin{tabular}[t]{||p{4cm}||}
        \hline
        Gianluca Nenz \\
        \hline\hline
        Meetings \\ 
        \hline
        Work 2 \\
        \hline
        Work 3 \\ 
        \hline
        Work 4\\[1ex] 
        \hline
    \end{tabular}
    \hfill
    \begin{tabular}[t]{||p{4cm}||}
        \hline
        Ronny Mueller \\
        \hline\hline
        Work 1 \\ 
        \hline
        Work 2 \\
        \hline
        Work 3 \\ 
        \hline
        Work 4\\[1ex] 
        \hline
    \end{tabular}
    \hfill
    \begin{tabular}[t]{||p{4cm}||}
        \hline
        Thomas Kleb \\
        \hline\hline
        Protocols \\ 
        \hline
        Documentation \\
        \hline
        Work 3 \\ 
        \hline
        Work 4\\[1ex] 
        \hline
    \end{tabular}
    \caption{Work Distribution per Student}
    \label{work_dis}
\end{table}

\subsection{Planning and Milestones}

\subsubsection*{Phases and Iterations}
The project is comprised of the four steps explained in \nameref{rup_section}. Each of those phases has multiple iterations which create the different sprints for the project. The meetings with the advisor will be on thursdays while the team meetings will be held tuesdays. Each iteration / sprint will be of a seven day length. \\

\noindent We started the "Studienarbeit" before we began with the regular school. In the week before we each made research and plans about the comming project. After having a talk with the advisor it was decided to first find out the level of knowledge each student has to make it easier for the advisor to plan.
\begin{table}[H]
    \centering
    \begin{tabular}{|p{0.1\linewidth}|p{0.15\linewidth}|p{0.15\linewidth}|p{0.46\linewidth}|}
        \hline
        \multicolumn{4}{||c||}{\textbf{Inception}} \\
        \hline \hline
        Iteration & Start & End & Description \\
        \hline \hline
        0 & 12.09.2022 & 18.09.2022 & Collection of Ideas and planning first meeting\\
        \hline
        1 & 19.09.2022 & 25.09.2022 & First meeting and handout of exercises to assess the knowledge of the students \\
        \hline
        2 & 26.09.2022 & 02.10.2022 & Working on the exercises and receiving solutions for harder ones \\
        \hline
    \end{tabular}
    \caption{RUP: Inception Phase Planning}
    \label{inception_table}
\end{table}

\noindent The elaboration phase is used to plan and assess the possible risks in this project. This consists of a documentation structure, the project plan and the risk management to make sure the construction phase has no major hickups.
\begin{table}[H]
    \centering
    \begin{tabular}{|p{0.1\linewidth}|p{0.15\linewidth}|p{0.15\linewidth}|p{0.46\linewidth}|}
        \hline
        \multicolumn{4}{||c||}{\textbf{Elaboration}} \\
        \hline \hline
        Iteration & Start & End & Description \\
        \hline \hline
        3 & 03.10.2022 & 09.10.2022 &  First big meeting with advisor; Creating project plan and risk analysis.\\
        \hline
        4 & 10.10.2022 & 13.10.2022 & Project Plan and Documentation is set; Problem Domains and Learningconcepts are defined \\
        \hline
        5 & 14.10.2022 & 25.10.2022 & Lab Conepts are defined \\
        \hline
    \end{tabular}
    \caption{RUP: Elaboration Phase Planning}
    \label{elaboration_table}
\end{table}

\noindent The construction phase is where the labs are primarily built. 
\begin{table}[H]
    \centering
    \begin{tabular}{|p{0.1\linewidth}|p{0.15\linewidth}|p{0.15\linewidth}|p{0.46\linewidth}|}
        \hline
        \multicolumn{4}{||c||}{\textbf{Construction}} \\
        \hline \hline
        Iteration & Start & End & Description \\
        \hline \hline
        6 & 25.10.2022 & 01.11.2022 & ----\\
        \hline
        7 & 01.11.2022 & 08.11.2022 & ---- \\
        \hline
        8 & 08.11.2022 & 15.11.2022 & ---- \\
        \hline
        9 & 15.11.2022 & 22.11.2022 & ---- \\
        \hline
        10 & 22.11.2022 & 29.11.2022 & ---- \\
        \hline
        11 & 29.11.2022 & 06.12.2022 & ---- \\
        \hline
    \end{tabular}
    \caption{RUP: Construction Phase Planning}
    \label{construction_table}
\end{table}

\noindent To make sure enough time is planned a buffer week was added to the transition phase. This phase is also mainly used to finish up the documentation and implement the different labs to Hacking Lab. The last week is used to clean up and hand in the documentation and abstract to both the OST and the advisor.
\begin{table}[H]
    \centering
    \begin{tabular}{|p{0.1\linewidth}|p{0.15\linewidth}|p{0.15\linewidth}|p{0.46\linewidth}|}
        \hline
        \multicolumn{4}{||c||}{\textbf{Transition}} \\
        \hline \hline
        Iteration & Start & End & Description \\
        \hline \hline
        12 & 06.12.2022 & 13.12.2022 & Buffer \\
        \hline
        13 & 13.12.2022 & 20.12.2022 & ---- \\
        \hline
        14 & 20.12.2022 & 23.12.2022 & ---- \\
        \hline
    \end{tabular}
    \caption{RUP: Transition Phase Planning}
    \label{transition_table}
\end{table}

\subsubsection*{Milestones}
To guarantee the success of the project milestones were defined with a deadline.

\begin{table}[H]
    \centering
    \begin{tabular}[]{|| p{5cm} | c | p{6.2cm} ||}
        \hline
        Milestones & Deadline & Description \\
        \hline \hline
        M1 - Solving RE Exercises & 05.10.2022 & The Team solves the given exercises to find the level of RE knowledge. \\
        \hline
        M2 - Defining Problem Domains and Learnconcepts& 13.10.2022 & Problem Domains are defined, first Lernconcepts are planned \\
        \hline
        M3 - Lab Concepts & 25.10.2022 & Lab Concepts are defined to start working on the construction. \\
        \hline
        M4 - Setup Labs & 06.12.2022 & Labs are setup and tested. \\
        \hline
        M5 - Hand-In & 23.12.2022 & Document is handed in to the advisor and OST \\
        \hline
    \end{tabular}
\end{table}

\subsubsection*{Time Tracking}
For time tracking the team has decided on using GitLabs integrated time tracking. 
\subsubsection*{Issue Tracking}
The issue tracking is done on GitLabs own interface to have as few difficulties as possible. To have an easier overview of the different issues the team has created tags to differentiate between the issues and their assigned student. 

\subsubsection*{Meetings}
The team has meetings each tuesday to elaborate problems and check up on the progress. This meetings are also used to distributed the work load and the different parts of the sprint. \\
On Thursdays the team meets the advisor Ivan Buetler to inform him on the progress done and the problems that came up. These meetings have different time schedules to fit everyones calender. \\
Each meeting will be documented and uploaded to the GIT repository. After each meeting the participants should know what to do and how to contact each other if any problems arise.
\subsubsection*{CI/CD}

\subsubsection*{Projectmanagment}
The whole project will use a GitLab repository. To make sure no confusion happens a multirepo principle is used where one repository is for the documentation and protocols only and other are for code, information gathered, etc. Each student works on a branch and before pushing to the main branch has another student look into the code / text written. 

\subsection{Testing}
This chapter describes how the quality of the challenges was determined and the feedback was evaluated. \\
The labs were tested by many different participants. This testing was done to ensure the challenges were solvable and understandable for the intended purpose. The testing can be divided into two parts.

\subsubsection{Internal Tests}
After every challenge was completed and uploaded to the Hacking-Lab platform, the creator checked the challenge again. After the creator finished his testing, he signaled the other students from the group to test the challenge. Any questions, mistakes or feedback from these tests were immediately reported to the creator. He then decided which parts he had to change. Additionally at the next supervisor meeting the challenge was presented and the supervisor was also able to give feedback to the challenge.

\subsubsection{Testing with third-party students}
In order to get realistic feedback from students, we asked cyber security students in their fifth semester to solve the labs and provide feedback. This feedback was collected using Google Forms. This tool allows to create simple surveys, which can then be analyzed in graphs automatically. \\
Only some of the challenges were solved by this group. The feedback received was helpful. However, since many of these students were also busy with a SA, it was a bit difficult for them to find the motivation to solve them thoroughly. Testing the labs meant a significant amount of work and therefore these tests were only done by a few individuals for all the challenges.

\subsubsection{Feedback}
The challenges were perceived as educating and easy to follow by the testing participants. The main takeaway from these tests were the following points:
\begin{itemize}
    \item Challenges are clear and interesting
    \item Challenges teach concepts in a fun way
    \item The time requirements for each challenge vary by much
  \end{itemize}

\subsubsection{Conclusion}
The feedback was mostly positive. Most of the feedback mentioned typing errors or some errors with setting up the challenges on Hacking-Lab.com. Some feedback also mentioned unprecise steps in the solutions. They were very useful to create the challenges in a way which should be understood by other students. \\
But we also think that our testing process could need some refining. We didn't really have a broad spectrum of knowledge and motivation in our testing participants because they had to do it in their leisure time. This way we only got very motivated students, who probably also had way more knowledge than the average one. In future projects we should strive to achieve a more normalized testing group by asking to be able to test some challenges in a security class or similar.


% --------------------------------------%
\section{Risk Analysis}
\chapter{Risks}
\section{Risk Managment}
For this project, the ”Project Management Triangle” is lacking the cost dimension, while the time dimension is fixed (strict deadlines). As a result, any risks that appear, automatically lead to a reduction of the project scope if there is no spare time. Because of this, we will prioritize dealing with risks above regular tasks and prioritize essential tasks over nice-to-haves, but we do not intend on planning in a flat time margin as we have no way to negotiate for more time.

\section{Estimated Risks}

\subparagraph{General Risks}
\begin{itemize}
    \item Finding Testing Participants (severity: medium, probability: high)\\ 
    \textbf{Mitigations:} Early looking for backup person \\ 
    \textbf{Actions taken:} Found backup person\\ 
    \textbf{New probability:} low
    \item Being able to create reverseable Programs with additional difficulties. (severity: very high, probability: medium)\\ 
    \textbf{Mitigations:} being able to ask Ivan \\ 
    \textbf{Actions taken:} Asked for possible help\\ 
    \textbf{New probability:} low
    \item Irreparable corruption of git server. (severity: very high, probability: low)\\ 
    \textbf{Mitigations:} Weekly off-site git server backups\\
    \textbf{Actions taken:} Repository mirrored to GitHub\\ 
    \textbf{New severity:} low
    \item Irreparable corruption of git server. (severity: very high, probability: low)\\ 
    \textbf{Mitigations:} Weekly off-site git server backups\\
    \textbf{Actions taken:} Repository mirrored to GitHub\\ 
    \textbf{New severity:} low
    \item Lost work due to un-pushed work. (severity: low, probability: high)\\ 
    \textbf{Mitigations:} Frequent reminders to push changes by Scrum Master / Team\\
\end{itemize}
\subparagraph{License Complications}
\begin{itemize}
    \item License Problems with Ghidra. (severity: high, probability: very low)\\ 
    \textbf{Mitigations:} No mitigations needed because it's completely Open Source.
    \item License Problems with IDA. (severity: high, probability: low)\\ 
    \textbf{Mitigations:} Providing a previously free version
\end{itemize}

\section{Project Monitoring}

\subsubsection*{Overview}

\subsubsection*{Milestones}

\subsubsection*{Time Tracking}

% --------------------------------------%
\section{Personal Rapports}

\subsubsection*{Gianluca Nenz}

\subsubsection*{Ronny Mueller}

\subsubsection*{Thomas Kleb}
