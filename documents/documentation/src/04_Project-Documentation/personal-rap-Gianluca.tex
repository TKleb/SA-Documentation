Thanks to Ivan we had a great start into this topic by solving sample challenges and thus evaluating our skill level.
At first we had to adjust our workflow since it differs from the one we have learnt in the "SEProject" previously.
Once everything settled we quickly got a good overview of the problem domain and started iteratively working on our labs which worked well. 
I really liked this approach and would love to do it again this way. 
For the next time though, I would plan some more ahead or have enough side projects ready to switch to if I finished the lab faster than anticipated.
Thanks to this I would always have enough to work on and my working hours would be distributed more evenly over the time of the project.
Finding willing testing participants turned out to be more difficult than we first thought because most students do not have enough spare time to test everything. \\
Thanks to our friends we were able to get everything tested but maybe we can somehow integrate testing as part of a lecture at OST or somewhere else in the future.  \\
All in all I am really happy with the outcome of our "Semesterarbeit". 
We have calculated enough time to finish up our labs and the documentation in time without the need of several nightshifts the days before hand in.
This could not have been made this smooth without the great teamwork we had. Thanks to everyone for putting effort into making good, beginner friendly labs. \\
I am thankful to have had the opportunity to bring reverse engineering into more students lives and I am looking forward to dig deeper in future projects.
