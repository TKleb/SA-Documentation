\label{sec:testing}
\subsection{Testing}
This chapter describes how the quality of the challenges and the feedback was evaluated.

\subsubsection{Introduction}
The labs were tested by different participants to ensure the challenges were solvable and understandable for the intended purpose. The testing can be divided into two parts.

\paragraph{Internal Tests}
After every challenge was completed and uploaded to the Hacking-Lab platform, they were tested by the team members which didn't create the challenge. This first feedback already fixed most of the big errors. The feedback resulting from these tests was usually immediately used to fix them in the challenge.

\paragraph{Testing third-party students}
The challenges also had to be tested by third party people who didn't have much knowledge about reverse engineering. In total 9 people tested the challenges and some of them gave their feedback in written form and some of them told the team directly. The feedback was then looked at in the sprint meetings and discussed which parts of the challenges had to change. These changes were also usually implemented in the timeframe of a week.

To have comparable feedback a Google Forms questionnaire was used.

\subsubsection{Feedback}
The challenges were perceived as educating and easy to follow by the testing participants.

The links to the written answers to our questionnaire can be found in the appendix.

\subsubsection{Conclusion}
The feedback was mostly positive. Most of the feedback mentioned typing errors or some errors with setting up the challenges on Hacking-Lab.com. Some feedback also mentioned unprecise steps in the solutions.

But we also think that our testing process could need some refining. We didn't really have a broad spectrum of knowledge and motivation in our testing participants because they had to do it in their free time. This way we only got very motivated students, who probably also had way more knowledge than the average one. In future projects we should strive to achieve a more normalized testing group by asking to be able to test some challenges in a security class or similar.
