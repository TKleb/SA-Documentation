Reverse Engineering for me was a relatively new subject but as I am highly interested in all things cybersecurity I was really glad, that we got accepted for this project. 
The start of it was a bit stressful for me since we never had to do a project this size without many rules and guidelines, we were totally free. \\
Once we had our first meetings with Ivan everything got clearer and we knew how the structure of this project is built. 
These meetings really helped me organize and I am extremly glad that we had such professional guideance from our advisor. 
Working on the different labs and finding problems to teach each student was my highlight and i am looking forward to do it again in my Bachelor Thesis. 
Our team was well organized and the different tasks were distributed fairly thanks to the multiple meetings which helped with the workflow.  
What I didnt plan for was the downtime between the creation of the labs. Which is why I want to plan more precisely in the future to have more work done in between.
The problems created during the building of the labs like docker not working as intended, LateX not doing what I want or even finding different test subjects, were a lesson to be taught at the end. 
For my next work I want to take what I have learned from this and improve on it, especially the preparation since we had to do some prepare work which could have been done beforehand. \\
At the end I am glad to have had the chance working on such a project and I am thankful for all the help we got from our advisor. 
Overall I enjoyed working in this team and on the project and can't wait to do it again next semester.
