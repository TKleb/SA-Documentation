\chapter{Technical Report}
\section{Introduction}
\subsection{Problem}
The subject of cybersecurity is constantly growing in importance for the computer scientist in general. The demand for cybersecurity experts with broad knowledge regarding current problems and malware is ever growing \cite{cybercrime-mag}. The Ostschweizer Fachhochschule (OST) has recognized this demand and has added more and more lectures for the cybersecurity interested students \cite{ost-cybersec}. One important aspect is still missing in the curriculum: reverse engineering.

\subsection{Similar Work}
To this date there is no work done regarding this subject in past projects. In the past, students have created different labs for the OST but none for reverse engineering. The infrastructure which is used for this project is already existing and established in the OST lectures. Hacking-Lab as the platform is known by the targeted student group and teachers alike. 

\subsection{Technologies Used}
To create each of the labs and their documentation multiple different tools and languages were used (shown in table \ref{tab:languages} and \ref{tab:tools}). 
\begin{center}
    \begin{table}[H]
        \centering
        \begin{tabular}{ |p{4.1cm}|p{10cm}| } 
            \hline
            \multicolumn{2}{||c||}{\textbf{Languages}} \\
            \hline
            \hline
                Assembler & As the base structure of a binary it was taught in multiple Courses before this. In the labs ASM is used to understand the flow of a function and what it does when executed. \\
            \hline
                C & All of the binaries were written in C.   \\
            \hline
                Python & As an easy to understand language Python is used to write the exploits after analyzing the binaries. \\
            \hline
        \end{tabular}
        \caption{Overview of all the languages used to create the labs.}
        \label{tab:languages}
    \end{table}
\end{center}

\begin{center}
    \begin{table}[H]
        \centering
        \begin{tabular}{ |p{4.1cm}|p{10cm}| } 
            \hline
            \multicolumn{2}{||c||}{\textbf{Tools}} \\
            \hline
            \hline
                VSCode & Each of the students of the project used VSCode for programming and documenting. This allowed for easier settings and more control of the output. \\
            \hline
                IDA Freeware & Each of the labs were tested before uploading to Hacking Lab. All of the tests were done in IDA Freeware since this software is used to show the solutions.  \\
            \hline
                Ghidra & Each of the labs were tested before uploading to Hacking Lab. Ghidra was used to check the pseudo code. This ensured that students using Ghidra instead of IDA Freeware have a solvable problem aswell and can follow the steps given. \\
            \hline
                HL Demo Tenant & 
                To test all of the labs, the demo tenant of Hacking Lab was used. This allowed for free testing without interfering with the OST tenant.  \\ 
            \hline
                Docker & Docker container can be used on the Hacking Lab platform to have a server side component to the challenges. \\
            \hline
                OST Gitlab & 
                To have versioning of the code OSTs Gitlab was used.  \\ 
            \hline
                Clockify & 
                This software allowed for time management. \\ 
            \hline
        \end{tabular}
        \caption{Overview of all the frameworks and tools used to create the labs.}
        \label{tab:tools}
    \end{table}
\end{center}

\subsection{Goals}
The goal for this project has multiple facets:
\begin{itemize}
    \item The creation of different labs to show the students of the OST the aspects of reverse engineering
    \item The students should have the following learning objectives:
    \begin{itemize}
        \item Gain an understanding of what reverse engineering is and what it can be used for
        \item Know the basic handling of debuggers and disassemblers
        \item Understand a binary programs control flow using static debugging
        \item Understand a binary programs control flow using dynamic debugging
        \item Be able to locate and exploit simple security flaws in a binary program.
    \end{itemize}
\end{itemize}

\subsection{Product}
Since every student uses his own machine with his own configurations, Hacking-Lab as a platform with its LiveCD was chosen as the foundation. This allows a simple, operating system independent approach to all excercises thanks to its webinterface and docker hosting capabilities. 

\section{Requirements for the Labs}

\subsection{Overview}
This chapter lists the different requirements we have defined in order to successfully create and solve the reverse engineering labs.
\subsection{Requirements}
\textbf{Language} \\
The labs and their solutions will be written in english to guarantee each student can understand it. \\[0.5cm]
\textbf{Targeted Group of Students} \\
The product of this project is aimed at students in their third year (fifth semester) or higher because it is an in-depth look at a cybersecurity subject. Students taking this course should have visited the mandatory subject "Cyber Security" to have basic information and maybe even "Secure Software" for more advanced knowledge of some exploitations. A rule of thumb is the more security lectures a student has visited and finished, the better. \\[0.5cm]
\textbf{Time Requirements} \\
Each of the labs has a different time requirement for the students. One lab should be solvable in an hour or less.  \\[0.5cm]
\textbf{Grading} \\
Depending on the lab, a student has to hand-in a flag and/or a writeup. These will be checked by the teacher or an automated system. \\[0.5cm]

\newpage
\section{Lab Documentation}
\subsection{Lab 1: Refresher}
\subsubsection*{Content}
Lorem Ipsum
\subsubsection*{Choice of Topic}
Lorem Ipsum
\subsubsection*{Objectives}
Lorem Ipsum
\subsubsection*{Grading}
Lorem Ipsum

\subsection{Lab 2: Introduction into Reverese Engineering}
\subsubsection*{Content}
Lorem Ipsum
\subsubsection*{Choice of Topic}
Lorem Ipsum
\subsubsection*{Objectives}
Lorem Ipsum
\subsubsection*{Grading}
Lorem Ipsum

\subsection{Lab 3: Static Debugging}
\subsubsection*{Content}
In this lab the student will learn how to use IDA Freeware to statically reverse a given binary. The goal is to find the main function of the both a normal binary and one with the symbols stripped. \\
This will confront the student with a problem that has to be solved and to do that the student has to understand the inner workings of the binary by reading the assembly code and not just relying on the GUI of the decompiling program.
\subsubsection*{Choice of Topic}
Program execution starts from the main() function which is why it is an important skill of a reverse engineer to find it and start understanding the rest of the binary from there on.
\subsubsection*{Objectives}
\begin{itemize}
    \item Find the main function in a normal binary
    \item Find the main function in a binary with symbols stripped
\end{itemize}
\subsubsection*{Grading}
The student has to inspect the main functions and find a flag in both of them. The final hand-in will be the combined flag.

\subsection{Lab 4: Dynamic Debugging}
\subsubsection*{Content}
In this lab the student will learn how to use GDB and x64dbg to dynamically reverse a given binary. In a first step the lab explains how to dynamically debug the binary given in Lab 3 and then, in a second step, the student will be presented with a new binary conataining a flag. \\
The goal of this lab is to show the student how this skill is used and then have him do it on his own to solidify the steps he has compleated before.
\subsubsection*{Choice of Topic}
Next to static debugging, dynamic debugging is used in a wide range of reverse engineering. Because of this it is important to have a student go through the steps and explain how it is done. 
\subsubsection*{Objectives}
\begin{itemize}
    \item Use GDB to find the flag of the binary
    \item Use x64dbg to find the flag of the binary
    \item Find the flag of a new binary using dynamic debugging
\end{itemize}
\subsubsection*{Grading}
The student has to use the newly aquired skills to find the flag in a new binary using dynamic debugging skills.

\subsection{Lab 5: First Reversing Attempts}
\subsubsection*{Content}
In this lab the student will deepen their knowledge of reversing binaries to find out how they work. This challenge contains 2 binaries in which you have a requirement which has to be met to recieve the flag.
\subsubsection*{Choice of Topic}
It is important to test the understanding of the programs used in an environment in which there is no step for step guide.
\subsubsection*{Objectives}
\begin{itemize}
    \item Use static or dynamic debugging to find the requirements of the programs
    \item Find out how program arguments are handled in assembly
\end{itemize}
\subsubsection*{Grading}
The student has to solve both binaries to get a flag

\subsection{Lab 6: Remote Login}
\subsubsection*{Content}
Lorem Ipsum
\subsubsection*{Choice of Topic}
Lorem Ipsum
\subsubsection*{Objectives}
Lorem Ipsum
\subsubsection*{Grading}
Lorem Ipsum

\subsection{Lab 7: Pwntools - Introduction}
\subsubsection*{Content}

\subsubsection*{Choice of Topic}

\subsubsection*{Objectives}
\begin{itemize}
    \item Use aquired skills to find vulnerability in binary
    \item Create a pwntools script to exploit that vulnerabilty
\end{itemize}
\subsubsection*{Grading}
Lorem Ipsum


% --------------------------------------%
\section{Risk Analysis}
\subsection{Risk Managment}
For this project, the ”Project Management Triangle” is lacking the cost dimension, while the time dimension is fixed (strict deadlines). As a result, any risks that appear, automatically lead to a reduction of the project scope if there is no spare time. Because of this, we will prioritize dealing with risks above regular tasks and prioritize essential tasks over nice-to-haves, but we do not intend on planning in a flat time margin as we have no way to negotiate for more time.

\subsection{Estimated Risks}
The numbers in the following picture describe the position of the risks on the following page from top to bottom.

\begin{figure}[H]
  \includegraphics[width=\linewidth]{resources/risks-matrix.png}
  \caption{Risk matrix}
  \label{risk_matrix}
\end{figure}

\begin{table}
  \centering
  \textbf{Risk 1:}
  \begin{tabular}{p{0.2\linewidth}p{0.8\linewidth}}
    Name            & Finding testing participants           \\
    Severity        & Medium                                 \\
    Probability     & High                                   \\
    Mitigations     & Already got confirmation from testers  \\
    New probability & Low                                   
  \end{tabular}
  \caption{Testing participants risk}
\end{table}

\begin{table}
  \centering
  \textbf{Risk 2:}
  \begin{tabular}{p{0.2\linewidth}p{0.8\linewidth}}
    Name            & Being able to create reversable programs with additional difficulties           \\
    Severity        & Very High                                 \\
    Probability     & Medium                                   \\
    Mitigations     & Assured the advisor is available for consultation  \\
    New probability & Low                                   
  \end{tabular}
  \caption{Create programs risk}
\end{table}

\begin{table}
  \centering
  \textbf{Risk 3:}
  \begin{tabular}{p{0.2\linewidth}p{0.8\linewidth}}
    Name            & Not enough time for the actual challenges because of too much programming etc.           \\
    Severity        & High                                 \\
    Probability     & Low                                   \\
    Mitigations     & Creating challenges in chronological order and in a iterating fashion  \\
    New probability & Very Low                                   
  \end{tabular}
  \caption{Not enough time risk}
\end{table}

\begin{table}
  \centering
  \textbf{Risk 4:}
  \begin{tabular}{p{0.2\linewidth}p{0.8\linewidth}}
    Name            & Irreparable corruption of git server           \\
    Severity        & Very High                                 \\
    Probability     & Low                                   \\
    Mitigations     & Weekly off-site git server backups  \\
    New severity    & Low                                   
  \end{tabular}
  \caption{Corruption of git risk}
\end{table}

\begin{table}
  \centering
  \textbf{Risk 5:}
  \begin{tabular}{p{0.2\linewidth}p{0.8\linewidth}}
    Name            & Lost work due to unpushed work           \\
    Severity        & Low                                 \\
    Probability     & High                                   \\
    Mitigations     & Frequent reminders to push changes  \\
    New probability & Low                                   
  \end{tabular}
  \caption{Unpushed work risk}
\end{table}

\begin{table}
  \centering
  \textbf{Risk 6:}
  \begin{tabular}{p{0.2\linewidth}p{0.8\linewidth}}
    Name            & License problems with used software           \\
    Severity        & High                                 \\
    Probability     & Low                                   \\
    Mitigations     & Trying to use opensource or public software  \\
    New probability & Low                                   
  \end{tabular}
  \caption{Licence problems risk}
\end{table}

