\chapter{Technical Report}
\section{Introduction}
\subsection{Problem}
The subject of cybersecurity is constantly growing in importance for the computer scientist in general. The demand for cybersecurity experts with broad knowledge regarding current problems and malware is ever growing \cite{cybercrime-mag}. The Ostschweizer Fachhochschule (OST) has recognized this demand and has added more and more lectures for the cybersecurity interested students \cite{ost-cybersec}. One important aspect is still missing in the curriculum: reverse engineering.

\subsection{Similar Work}
To this date there is no work done regarding this subject in past projects. In the past, students have created different labs for the OST but none for reverse engineering. The infrastructure which is used for this project is already existing and established in the OST lectures. Hacking-Lab as the platform is known by the targeted student group and teachers alike. 

\subsection{Technologies Used}
To create each of the labs and their documentation multiple different tools and languages were used (shown in table \ref{tab:languages} and \ref{tab:tools}). 
\begin{center}
    \begin{table}[H]
        \centering
        \begin{tabular}{ |p{4.1cm}|p{10cm}| } 
            \hline
            \multicolumn{2}{||c||}{\textbf{Languages}} \\
            \hline
            \hline
                Assembler & As the base structure of a binary it was taught in multiple Courses before this. In the labs ASM is used to understand the flow of a function and what it does when executed. \\
            \hline
                C & All of the binaries were written in C.   \\
            \hline
                Python & As an easy to understand language Python is used to write the exploits after analyzing the binaries. \\
            \hline
        \end{tabular}
        \caption{Overview of all the languages used to create the labs.}
        \label{tab:languages}
    \end{table}
\end{center}

\begin{center}
    \begin{table}[H]
        \centering
        \begin{tabular}{ |p{4.1cm}|p{10cm}| } 
            \hline
            \multicolumn{2}{||c||}{\textbf{Tools}} \\
            \hline
            \hline
                VSCode & Each of the students of the project used VSCode for programming and documenting. This allowed for easier settings and more control of the output. \\
            \hline
                IDA Freeware & Each of the labs were tested before uploading to Hacking-Lab. All of the tests were done in IDA Freeware since this software is used to show the solutions.  \\
            \hline
                Ghidra & Each of the labs were tested before uploading to Hacking-Lab. Ghidra was used to check the pseudo code. This ensured that students using Ghidra instead of IDA Freeware have a solvable problem aswell and can follow the steps given. \\
            \hline
                HL Demo Tenant & 
                To test all of the labs, the demo tenant of Hacking-Lab was used. This allowed for free testing without interfering with the OST tenant.  \\ 
            \hline
                Docker & Docker container can be used on the Hacking-Lab platform to have a server side component to the challenges. \\
            \hline
                OST GitLab & 
                To have versioning of the code OSTs GitLab was used.  \\ 
            \hline
                Clockify & 
                This software allowed for time management. \\ 
            \hline
        \end{tabular}
        \caption{Overview of all the frameworks and tools used to create the labs.}
        \label{tab:tools}
    \end{table}
\end{center}

\subsection{Goals}
The goal for this project has multiple facets:
\begin{itemize}
    \item The creation of different labs to show the students of the OST the aspects of reverse engineering
    \item The students should have the following learning objectives:
    \begin{itemize}
        \item Gain an understanding of what reverse engineering is and what it can be used for
        \item Know the basic handling of debuggers and disassemblers
        \item Understand a binary programs control flow using static debugging
        \item Understand a binary programs control flow using dynamic debugging
        \item Be able to locate and exploit simple security flaws in a binary program.
    \end{itemize}
\end{itemize}

\subsection{Product}
Since every student uses his own machine with his own configurations, Hacking-Lab as a platform with its LiveCD was chosen as the foundation. This allows a simple, operating system independent approach to all excercises thanks to its webinterface and docker hosting capabilities. 

\section{Requirements for the Labs}

\subsection{Overview}
This chapter lists the different requirements we have defined in order to successfully create and solve the reverse engineering labs.
\subsection{Requirements}
\textbf{Language} \\
The labs and their solutions will be written in english to guarantee each student can understand it. \\[0.5cm]
\textbf{Targeted Group of Students} \\
The product of this project is aimed at students in their third year (fifth semester) or higher because it is an in-depth look at a cybersecurity subject. Students taking this course should have visited the mandatory subject "Cyber Security" to have basic information and maybe even "Secure Software" for more advanced knowledge of some exploitations. A rule of thumb is the more security lectures a student has visited and finished, the better. \\[0.5cm]
\textbf{Time Requirements} \\
Each of the labs has a different time requirement for the students. One lab should be solvable in an hour or less.  \\[0.5cm]
\textbf{Grading} \\
Depending on the lab, a student has to hand-in a flag and/or a writeup. These will be checked by the teacher or an automated system. \\[0.5cm]




