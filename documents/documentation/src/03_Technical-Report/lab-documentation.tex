\subsection{Lab 1: Refresher}
\subsubsection*{Content}
Lorem Ipsum
\subsubsection*{Choice of Topic}
Lorem Ipsum
\subsubsection*{Objectives}
Lorem Ipsum
\subsubsection*{Grading}
Lorem Ipsum

\subsection{Lab 2: Introduction into Reverese Engineering}
\subsubsection*{Content}
Lorem Ipsum
\subsubsection*{Choice of Topic}
Lorem Ipsum
\subsubsection*{Objectives}
Lorem Ipsum
\subsubsection*{Grading}
Lorem Ipsum

\subsection{Lab 3: Static Debugging}
\subsubsection*{Content}
In this lab the student will learn how to use IDA Freeware to statically reverse a given binary. The goal is to find the main function of the both a normal binary and one with the symbols stripped. \\
This will confront the student with a problem that has to be solved and to do that the student has to understand the inner workings of the binary by reading the assembly code and not just relying on the GUI of the decompiling program.
\subsubsection*{Choice of Topic}
Program execution starts from the main() function which is why it is an important skill of a reverse engineer to find it and start understanding the rest of the binary from there on.
\subsubsection*{Objectives}
\begin{itemize}
    \item Find the main function in a normal binary
    \item Find the main function in a binary with symbols stripped
\end{itemize}
\subsubsection*{Grading}
The student has to inspect the main functions and find a flag in both of them. The final hand-in will be the combined flag.

\subsection{Lab 4: Dynamic Debugging}
\subsubsection*{Content}
In this lab the student will learn how to use GDB and x64dbg to dynamically reverse a given binary. In a first step the lab explains how to dynamically debug the binary given in Lab 3 and then, in a second step, the student will be presented with a new binary conataining a flag. \\
The goal of this lab is to show the student how this skill is used and then have him do it on his own to solidify the steps he has compleated before.
\subsubsection*{Choice of Topic}
Next to static debugging, dynamic debugging is used in a wide range of reverse engineering. Because of this it is important to have a student go through the steps and explain how it is done. 
\subsubsection*{Objectives}
\begin{itemize}
    \item Use GDB to find the flag of the binary
    \item Use x64dbg to find the flag of the binary
    \item Find the flag of a new binary using dynamic debugging
\end{itemize}
\subsubsection*{Grading}
The student has to use the newly aquired skills to find the flag in a new binary using dynamic debugging skills.

\subsection{Lab 5: First Reversing Attempts}
\subsubsection*{Content}
Lorem Ipsum
\subsubsection*{Choice of Topic}
Lorem Ipsum
\subsubsection*{Objectives}
Lorem Ipsum
\subsubsection*{Grading}
Lorem Ipsum

\subsection{Lab 6: Remote Login}
\subsubsection*{Content}
Lorem Ipsum
\subsubsection*{Choice of Topic}
Lorem Ipsum
\subsubsection*{Objectives}
Lorem Ipsum
\subsubsection*{Grading}
Lorem Ipsum

\subsection{Lab 7: Pwntools - Introduction}
\subsubsection*{Content}

\subsubsection*{Choice of Topic}

\subsubsection*{Objectives}
\begin{itemize}
    \item Use aquired skills to find vulnerability in binary
    \item Create a pwntools script to exploit that vulnerabilty
\end{itemize}
\subsubsection*{Grading}
Lorem Ipsum