\chapter{Abstract}
\subsection{Background}
As it stands today there is no module or learning unit at the ost to teach the concept of reverse engineering software. 

\subsection{Purpose}
This should change. Goal of this "Studienarbeit" is to create challenges in the topic of reverse engineering. These challenges will be used by lecturers of the OST in Rapperswil-Jona to teach the basics to the students.

\subsection{Methods}
To achieve this goal we created challenges on the Hacking-Lab platform in ascending difficulty, which are teaching the basics and then also some more advanced concepts of bugs and exploits. To ensure the quality of our challenges we did organise testing participants, who are in our target demographic.

\subsection{Results}
The goals were to create a total of up to 10 challenges in this "Studienarbeit". This goal was achieved with a total of 11 challenges created. The testing was also successfully carried out and was overall positive. If there was some common feedback, we adjusted the corresponding challenge based on it. The chapter [section \ref{sec:task}] covers this in detail.

\subsection{Conclusions}
This project was overall very successfull, but there has also to be said that the topic of reverse engineering is huge and only a very small portion of it was covered. There are many more techniques and tools which have not been covered yet and this could be a future work.

