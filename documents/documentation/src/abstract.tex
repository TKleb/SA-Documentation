\chapter{Abstract}

\subsection*{Background}
As it stands today there is no module or learning unit at the OST to teach the concept of reverse engineering software.

\subsection*{Purpose}
Because reverse engineering is an important topic in the cyber security space, the goal of this SA is to create challenges in the topic of reverse engineering. These challenges will or can be used by lecturers of the OST in Rapperswil-Jona to teach the basics to the students.

\subsection*{Methods}
To achieve this goal we created challenges in ascending difficulty. At first the students will get information about the software they will use and the overall strategies of reverse engineering. After that they learn about some more advanced concepts. The focus here is creating the challenges in a manner, which should teach the basics in a easy to understand fashion. These challenges will be hosted on the Hacking-Lab platform. To ensure the quality of our challenges we did organise testing participants, who are cyber security students in their fifth semester as well. These tests should be the indicator if the goal was reached.

\subsection*{Results}
The primary goal was to create a total of up to 10 challenges in this SA. This goal was achieved with a total of 11 created challenges. The testing was also successfully carried out and the feedback overall positive. If there was some common feedback, we adjusted the corresponding challenge based on it. The chapter [section \ref{sec:testing}] covers this in detail.

\subsection*{Conclusions}
This project was overall very successful, but there has also to be said that the topic of reverse engineering is huge and only a very small portion of it was covered in these challenges. There are many more techniques and tools which have not been covered yet and this could be a future work.

