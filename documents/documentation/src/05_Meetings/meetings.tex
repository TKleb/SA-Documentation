\chapter*{Meetings}
\addcontentsline{toc}{chapter}{Meetings}
\label{meeting_chapter}

\begin{table}[H]
    \centering
    \begin{tabular}{| c | c | c | p{0.36\linewidth} | c |}
        \hline 
        Nr & Phase & Date & Description & Duration [min]\\
        \hline \hline
        1 & Elaboration & 06.10.2022 & Coordinate the project, documentation \- and ideas & 90\\
        \hline
        2 & Elaboration & 13.10.2022 & Present the Problemdomain with Learning Concepts and define the project plan & 60 \\
        \hline
        3 & Elaboration & 20.10.2022 & Lab Concept Drafts, GANTT Diagram & 80 \\
        \hline
        4 & Construction & 27.10.2022 & Think about the exploitation aspect and add it to mindmap; POC for Lab 2 and 3 and started testing & 86\\
        \hline
        5 & Construction & 03.11.2022 & POC for Lab running with Docker , Finish Lab 5 & 80 \\
        \hline
        6 & Construction & 10.11.2022 & Finish Labs and Create a POC for pwntools lab & 120 \\
        \hline
        7 & Construction & 17.11.2022 & Fix Pwntools lab, Introduction labs, Testing & 60 \\
        \hline
        8 & Construction & 25.11.2022 & Concept for Patching Lab, Testing, Refresher Lab & 90 \\
        \hline
        9 & Construction & 01.12.2022 & Patching Lab Fixing, Testing, Setting up Forms for Feedback & 60 \\
        \hline
        10 & Transition & 13.12.2022 & Discussing Documentation & 120 \\
        \hline
    \end{tabular}
    \caption{Meetings held with advisor}
    \label{meetings_hold_table}
\end{table}

\newpage
\section*{Elaboration Phase}
% ---------------------------------------------------- NEXT MEETING --------------------------------------------------------------------------%
\subsection*{Meeting 1: 06. Okt.}
\textbf{Deliverables:} Draft the Problemdomains together with Learningconcepts and the Projectplan. \\
\textbf{Discussed Topics:} 
\begin{itemize}
    \item Documentation
    \item Goal of the Project: Hacking Lab Integration for cyber security classes.
    \begin{itemize}
        \item Teacher needs required knowhow
        \item Labs built for a student attending the third year
    \end{itemize}
    \item Brainstorming for Problemdomains
    \begin{itemize}
        \item Learningconcepts need to be defined for teacher and for Lab construction
        \item RE Tools
        \item Required Knowhow for a Reverse Engineer
    \end{itemize}
    \item Mini Projectplan Brainstorming:
    \begin{itemize}
        \item Analysis of Problemdomains
        \item Setting up Learningconcepts
        \item What is important?
        \item Build Lab Concepts
        \item Implement with Tests
    \end{itemize}
\end{itemize}
\textbf{Decisions:} 
\begin{itemize}
    \item Idea of the project: Hacking Lab Integration (Modulintegration)
\end{itemize}
\textbf{Duration:} 1h 30min

\newpage
% ---------------------------------------------------- NEXT MEETING --------------------------------------------------------------------------%
\subsection*{Meeting 2: 13. Okt.}
\textbf{Deliverables:} Build some Lab Concepts (deliverables and overall subject), GANTT Diagramm \\
\textbf{Discussed Topics:} 
\begin{itemize}
    \item Problemdomain Mindmap
    \item Learning Concepts
    \item What a Hacking-Lab course needs
    \begin{itemize}
        \item Deliverables defined
        \item Show which skills are trained (Mindmap / MITRA style table)
        \item Exercises created as Markdown for ease of use 
        \item Get inspiration from already established courses
    \end{itemize}
\end{itemize}
\textbf{Decisions:} 
\begin{itemize}
    \item Add passive and active labs for networks.
    \item Write protocol of sprint meetings for own use but not needed in documentation.
    \item Create Hacking-Lab content on markdown files
\end{itemize} 
\textbf{Duration:} 1h

\newpage
% ---------------------------------------------------- NEXT MEETING --------------------------------------------------------------------------%
\subsection*{Meeting 3: 20. Okt.}
\textbf{Deliverables:} 
\begin{itemize}
    \item Students:
    \begin{itemize}
        \item Think about the exploitation aspect and add it to mindmap.
        \item POC for Lab 2 and 3 and started testing.
        \item Update GANTT chart.
    \end{itemize}
    \item Advisor:
    \begin{itemize}
        \item Sample Courses for RE.
        \item Unlock ECSC Hacking-Labs. 
    \end{itemize}
\end{itemize}
\textbf{Discussed Topics:} 
\begin{itemize}
    \item Online GANTT chart instead of Excel chart
    \item Refresher and Lab 1 (Introduction \#1)
    \begin{itemize}
        \item Create at the end to know what is necessary for the labs
        \item Use binaries from the already created labs
    \end{itemize}
    \item Server part is important (analyzing and exploiting remote); socat discussed
    \item Generally try to add exploits at the end
    \item RE is used to find weaknesses and exploit them
    \item Markdown for the Hacking-Lab consists of: Main page / Step pages and a solution page
\end{itemize}
\textbf{Decisions:}
\begin{itemize}
    \item Timetracking: Only if the used tool (clockify) allows it, add estimated time
    \item Plan Labs while creating others to maximize the efficency
    \item Start construction phase now (21.10.2022)
    \item Add exploits to the labs (remote server using docker)
    \item Start construction with lab 2 and 3 to ensure the refresher and lab 1 are done correctly
\end{itemize}
\textbf{Duration:} 1h 20min

\newpage
% ---------------------------------------------------- NEXT MEETING --------------------------------------------------------------------------%
\section*{Construction}
\subsection*{Meeting 4: 27. Okt.}
\textbf{Deliverables:} 
\begin{itemize}
    \item Students:
    \begin{itemize}
        \item POC for Lab running inside a Docker with downloadable binary and generated flag.
        \item POC for RE-Lab with different compiler using public code.
    \end{itemize}
    \item Advisor:
    \begin{itemize}
        \item Update Hacking Lab CD with Packet for IDA.
        \item Upload Introduction labs 2 and 3 to Hacking-Lab.
    \end{itemize}
\end{itemize}
\textbf{Discussed Topics:} 
\begin{itemize}
    \item S6 Overlay
    \item Use Hacking Lab Github Repo for References
\end{itemize}
\textbf{Decisions:} 
\begin{itemize}
    \item Windows: Bootstrap script for Windows (choco)
    \item Linux: Hacking Lab CD Packet for IDA
    \item Use Source Code of Example Exercises for Ideas
\end{itemize}
\textbf{Duration:} 1h 30min

\newpage
% ---------------------------------------------------- NEXT MEETING --------------------------------------------------------------------------%
\subsection*{Meeting 5: 03. Nov.}
\textbf{Deliverables:}
\begin{itemize}
    \item Students:
    \begin{itemize}
        \item Finish Remote-Exploit Lab
        \item Check out sent Labs and Links
        \item Finish RE-Attempts Lab
        \item POC für Python PWN Lab
        \item IDA Installation Guide
    \end{itemize}
    \item Advisor:
    \begin{itemize}
        \item Upload Finished Labs to Hacking-Lab
    \end{itemize}
\end{itemize} 
\textbf{Discussed Topics:}
\begin{itemize}
    \item How a Hacking-Lab Lab is build (Sections and Steps) and how the "Compe\-titive"-mode works
    \item Adding Security Questions as second section
    \item Add to the Labs what the student should hand in (Flag, Writeup, Multiple-Choice or a combination of the three)
    \item Update of the two Labs (Remote and RE-Attempts)
    \item Hacking-Lab Generator ("generator-hl-challenge" on GitHub)
\end{itemize}
\textbf{Decisions:}
\begin{itemize}
    \item Add difficulty to RE-Attempts Lab
    \item Don't use pyinstaller for packing since it makes the reversing annoying
    \item Create a PWN Library Lab
    \item Send PDF a day before meeting because of CICD
    \item Add docker-compose to all Docker folders
    \item Choose Mode to run the lab in (Competitive, Training, Optional steps)
    \item Use Section, Text and Step for Markdown
\end{itemize}
\textbf{Duration:} 1h 20min

\newpage
% ---------------------------------------------------- NEXT MEETING --------------------------------------------------------------------------%
\subsection*{Meeting 6: 10. Nov.}
\textbf{Deliverables:}
\begin{itemize}
    \item Students:
    \begin{itemize}
        \item Fix Pwntool Lab
        \item Introduction Labs for IDA, GDB and x64dbg
        \item Test the newly uploaded labs
    \end{itemize}
    \item Advisor:
    \begin{itemize}
        \item Migrate Challenges from OST to Demo tenant
    \end{itemize}
\end{itemize} 
\textbf{Discussed Topics:}
\begin{itemize}
    \item Demo tenant for easier access and testing
    \item Hacking Lab Tutorial with editor and resource pages
    \item Reversing is used to understand code and explotiation is needed too.
    \item Lab ideas (IDA scripting, Virus lab, more Client-Server labs)
\end{itemize}
\textbf{Decisions:}
\begin{itemize}
    \item Use demo tenant for labs before uploading to OST tenant    
\end{itemize}
\textbf{Duration:} 2h

\newpage
% ---------------------------------------------------- NEXT MEETING --------------------------------------------------------------------------%
\subsection*{Meeting 7: 17. Nov.}
\textbf{Deliverables:}
\begin{itemize}
    \item Students:
    \begin{itemize}
        \item Testing Labs
        \item GANTT-Diagram
        \item Refresher Lab
        \item Concept for the last labs
    \end{itemize}
    \item Advisor:
    \begin{itemize}
        \item Fix Modes and Settings for labs
    \end{itemize}
\end{itemize} 
\textbf{Discussed Topics:}
\begin{itemize}
    \item IDA installation on Hacking-Lab CD
    \item How to structure the Labs to upload it easily
    \item Other tenant for easier updating
    \item Pwntools lab problems with docker
    \item Refresher lab with coding Example
    \item More ideas for last lab: PWN iter (Bruteforcing), DLL Virus, Patching lab, reverse shell
\end{itemize}
\textbf{Decisions:}
\begin{itemize}
    \item Introduction structure with different steps
    \item Read Mail regarding documentation structure
    \item Use given structure for the documentation
    \item Create a crypto lab
    \item Concepts for the last lab
\end{itemize}
\textbf{Duration:} 2h

\newpage
% ---------------------------------------------------- NEXT MEETING --------------------------------------------------------------------------%
\subsection*{Meeting 8: 25. Nov.}
\textbf{Deliverables:}
\begin{itemize}
    \item Students:
    \begin{itemize}
        \item Patching Lab
        \item Testing
        \item Mail for Pwntools lab
    \end{itemize}
    \item Advisor:
    \begin{itemize}
        \item Upload Pwntools lab
    \end{itemize}
\end{itemize} 
\textbf{Discussed Topics:}
\begin{itemize}
    \item Pwntools lab
    \item Cryptographic lab works and is uploaded
    \item Assembly refresher is finished and uploaded
    \item Patching lab concept as favorite
    \item Calculator which needs to be patched to works
    \item Use website to upload patching file and display output
\end{itemize}
\textbf{Decisions:}
\begin{itemize}
    \item Patching lab 
    \item Start first testing
    \item Update documentation
\end{itemize}
\textbf{Duration:} 2h

\newpage
% ---------------------------------------------------- NEXT MEETING --------------------------------------------------------------------------%
\subsection*{Meeting 9: 1.Dez.}
\textbf{Deliverables:}
\begin{itemize}
    \item Students:
    \begin{itemize}
        \item Patching Lab finished 
        \item Feedback Forms setup and integrated
    \end{itemize}
\end{itemize} 
\textbf{Discussed Topics:}
\begin{itemize}
    \item Patching lab status 
    \item Testing setup and forms
    \item Overview of all labs
    \item Documentation status
    \item Next meetings
\end{itemize}
\textbf{Decisions:}
\begin{itemize}
    \item Only one more meeting (13. Dez.)
    \item All labs are ready and can get tested
    \item Finalizing documentation until next meeting
    \item Use structure given by OST for the documentation
\end{itemize}
\textbf{Duration:} 1h

\newpage
\section*{Transition}
% ---------------------------------------------------- NEXT MEETING --------------------------------------------------------------------------%
\subsection*{Meeting 10: 13.Dez.}
\textbf{Deliverables:}
\begin{itemize}
    \item Students:
    \begin{itemize}
        \item -
    \end{itemize}
\end{itemize} 
\textbf{Discussed Topics:}
\begin{itemize}
    \item What belongs into Management-Summary and how to write it
    \item What to cover in technical report
\end{itemize}
\textbf{Decisions:}
\begin{itemize}
    \item Management-Summary should be written in a manner that someone with zero knowledge would understand our project
    \item Technical report can be written in a manner that professionals should be able to understand it
\end{itemize}
\textbf{Duration:} 30min
