\chapter{Meetings}
\label{meeting_chapter}

\begin{table}[H]
    \centering
    \begin{tabular}{| c | c | c | p{0.36\linewidth} | c |}
        \hline 
        Nr & Phase & Date & Description & Duration [min]\\
        \hline \hline
        1 & Elaboration & 06.10.2022 & Coordinate the project, documentation \- and ideas & 90\\
        \hline
        2 & Elaboration & 13.10.2022 & Present the Problemdomain with Learning Concepts and define the project plan & 60 \\
        \hline
        3 & Elaboration & 20.10.2022 & Lab Concept Drafts, GANTT Diagram & 80 \\
        \hline
        4 & Construction & 27.10.2022 & Think about the exploitation aspect and add it to mindmap; POC for Lab 2 and 3 and started testing & 86\\
        \hline
        5 & Construction & 03.11.2022 & POC for Lab running with Docker , Finish Lab 5 & 80 \\
        \hline
        6 & Construction & 10.11.2022 & Finish Labs and Create a POC for pwntools lab & 120 \\
        \hline
    \end{tabular}
    \caption{Meetings held with advisor}
    \label{meetings_hold_table}
\end{table}
\section{Elaboration}
% ---------------------------------------------------- NEXT MEETING --------------------------------------------------------------------------%
\subsection*{Meeting 1: 06. Okt.}
\textbf{Deliverables:} Draft the Problemdomains together with Learningconcepts and the Projectplan. \\
\textbf{Discussed Topics:} 
\begin{itemize}
    \item Documentation
    \item Goal of the Project: Hacking Lab Integration for cyber security classes.
    \begin{itemize}
        \item Teacher needs required knowhow
        \item Labs built for a student attending the third year
    \end{itemize}
    \item Brainstorming for Problemdomains
    \begin{itemize}
        \item Learningconcepts need to be defined for teacher and for Lab construction
        \item RE Tools
        \item Required Knowhow for a Reverse Engineer
    \end{itemize}
    \item Mini Projectplan Brainstorming:
    \begin{itemize}
        \item Analysis of Problemdomains
        \item Setting up Learningconcepts
        \item What is important?
        \item Build Lab Concepts
        \item Implement with Tests
    \end{itemize}
\end{itemize}
\textbf{Decisions:} 
\begin{itemize}
    \item Idea of the project: Hacking Lab Integration (Modulintegration)
\end{itemize}
\textbf{Duration:} 1 hour 30 min
% ---------------------------------------------------- NEXT MEETING --------------------------------------------------------------------------%
\subsection*{Meeting 2: 13. Okt.}
\textbf{Deliverables:} Build some Lab Concepts (deliverables and overall subject), GANTT Diagramm \\
\textbf{Discussed Topics:} 
\begin{itemize}
    \item Problemdomain Mindmap
    \item Learning Concepts
    \item What a Hacking-Lab course needs
    \begin{itemize}
        \item Deliverables defined
        \item Show which skills are trained (Mindmap / MITRA style table)
        \item Exercises created as Markdown for ease of use 
        \item Get inspiration from already established courses
    \end{itemize}
\end{itemize}
\textbf{Decisions:} 
\begin{itemize}
    \item Add passive and active labs for networks.
    \item Write protocol of sprint meetings for own use but not needed in documentation.
    \item Create Hacking-Lab content on markdown files
\end{itemize} 
\textbf{Duration:} 1 hour
% ---------------------------------------------------- NEXT MEETING --------------------------------------------------------------------------%
\subsection*{Meeting 3: 20. Okt.}
\textbf{Deliverables:} 
\begin{itemize}
    \item Students:
    \begin{itemize}
        \item Think about the exploitation aspect and add it to mindmap.
        \item POC for Lab 2 and 3 and started testing.
        \item Update GANTT chart.
    \end{itemize}
    \item Advisor:
    \begin{itemize}
        \item Sample Courses for RE.
        \item Unlock ECSC Hacking-Labs. 
    \end{itemize}
\end{itemize}
\textbf{Discussed Topics:} 
\begin{itemize}
    \item Online GANTT chart instead of Excel chart
    \item Refresher and Lab 1 (Introduction \#1)
    \begin{itemize}
        \item Create at the end to know what is necessary for the labs
        \item Use binaries from the already created labs
    \end{itemize}
    \item Server part is important (analyzing and exploiting remote); socat discussed
    \item Generally try to add exploits at the end
    \item RE is used to find weaknesses and exploit them
    \item Markdown for the Hacking-Lab consists of: Main page / Step pages and a solution page
\end{itemize}
\textbf{Decisions:}
\begin{itemize}
    \item Timetracking: Only if the used tool (clockify) allows it, add estimated time
    \item Plan Labs while creating others to maximize the efficency
    \item Start construction phase now (21.10.2022)
    \item Add exploits to the labs (remote server using docker)
    \item Start construction with lab 2 and 3 to ensure the refresher and lab 1 are done correctly
\end{itemize}
\textbf{Duration:} 1 hour 20min
% ---------------------------------------------------- NEXT MEETING --------------------------------------------------------------------------%
\section{Construction}
\subsection*{Meeting 4: 27. Okt.}
\textbf{Deliverables:} 
\begin{itemize}
    \item Students:
    \begin{itemize}
        \item POC for Lab running inside a Docker with downloadable binary and generated flag.
        \item POC for RE-Lab with different compiler using public code.
    \end{itemize}
    \item Advisor:
    \begin{itemize}
        \item Update Hacking Lab CD with Packet for IDA.
        \item Upload Introduction labs 2 and 3 to Hacking-Lab.
    \end{itemize}
\end{itemize}
\textbf{Discussed Topics:} 
\begin{itemize}
    \item S6 Overlay
    \item Use Hacking Lab Github Repo for References
\end{itemize}
\textbf{Decisions:} 
\begin{itemize}
    \item Windows: Bootstrap script for Windows (choco)
    \item Linux: Hacking Lab CD Packet for IDA
    \item Use Source Code of Example Exercises for Ideas
\end{itemize}
\textbf{Duration:} 1h 30min
% ---------------------------------------------------- NEXT MEETING --------------------------------------------------------------------------%
\subsection*{Meeting 5: 03. Nov.}
\textbf{Deliverables:}
\begin{itemize}
    \item Students:
    \begin{itemize}
        \item Finish Remote-Exploit Lab
        \item Check out sent Labs and Links
        \item Finish RE-Attempts Lab
        \item POC für Python PWN Lab
        \item IDA Installation Guide
    \end{itemize}
    \item Advisor:
    \begin{itemize}
        \item Upload Finished Labs to Hacking-Lab
    \end{itemize}
\end{itemize} 
\textbf{Discussed Topics:}
\begin{itemize}
    \item How a Hacking-Lab Lab is build (Sections and Steps) and how the "Compe\-titive"-mode works
    \item Adding Security Questions as second section
    \item Add to the Labs what the student should hand in (Flag, Writeup, Multiple-Choice or a combination of the three)
    \item Update of the two Labs (Remote and RE-Attempts)
    \item Hacking-Lab Generator ("generator-hl-challenge" on GitHub)
\end{itemize}
\textbf{Decisions:}
\begin{itemize}
    \item Add difficulty to RE-Attempts Lab
    \item Don't use pyinstaller for packing since it makes the reversing annoying
    \item Create a PWN Library Lab
    \item Send PDF a day before meeting because of CICD
\end{itemize}
\textbf{Duration:} 1h 20min
% ---------------------------------------------------- NEXT MEETING --------------------------------------------------------------------------%
\subsection*{Meeting 6: 10. Nov.}
\textbf{Deliverables:}
\begin{itemize}
    \item Students:
    \begin{itemize}
        \item Remote Exploit Lab Finished
        \item Markdown file structure 
        \item RE-Attempts: no use for pyinstaller and fix programs
        \item POC for pwntools lab
        \item Check out old mails
    \end{itemize}
\end{itemize} 
\textbf{Discussed Topics:}
\begin{itemize}
    \item IDA installation on Hacking-Lab CD
    \item How to structure the Labs to upload it easily
    \item Other tenant for easier updating
    \item pwntools lab problems with docker
\end{itemize}
\textbf{Decisions:}
\begin{itemize}
    \item Use demo tenant for labs 
    \item Have lab in folders of section, steps and text
    \item Choose Mode to run the lab in (Competitive, Training, Optional steps)
    \item Dont use pyinstaller in RE-Attempts
    \item Add docker-compose to all Docker folders
\end{itemize}
\textbf{Duration:} 2h

\section{Transition}
