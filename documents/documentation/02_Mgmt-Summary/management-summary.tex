\chapter{Management Summary}
\section{Overview}
\subsection{What is Reverse Engineering and what is it used for}
Reverse engineering is the process or method through which a person tries to understand the inner workings of a device, process, system or software to further understand how it accomplishes tasks. In basic terms it is the process of disassembling or dissecting an object to see how it works and find out ways to improve it or exploit a found error. In this project we focus on its aspects in software - and computer engineering. In these sections of science, reverse engineering is used by multiple parties to find mistakes in the software's code or its execution and exploit or fix them. 

\subsection{Current Situation}
For a computer scientist it is always useful to have knowledge in cyber security subjects. To accomplish the task of showing the students the world of cyber security, the "Ostschweizer Fachhochschule" (OST) implemented several modules like "Cyber Security", "Secure Software" and the newest one "Cyber Defense". In these modules the professors explained the different acpects using Hacking-Lab as platform for the exercises. The plan is to extend the current state with reverse engineering labs and exercises. The goal of them is to bring the students closer to this subject and explain the fundamentals of reverse engineering.

\subsection{Approach}
To achieve these tasks, new exercises will be added to Hacking-Lab OST environment, which is, as mentioned above, a platform with which students are already accustomed to. These exercises will be added in form of labs for the sutent to go through and will be built with the idea of future additions in mind.

\section{Procedure}
At the start of the project, the scope was defined, based on which the labs are created. This scope includes the difficulty increase between each lab, the knowhow to be tought to understand the procedure and which tools are used by the student to finish the tasks. In addition to these points a platform on which the student is intended to work on is defined.
For the students to solve the given tasks they needed a provided infrastructure to follow. Hacking-Lab is used to host the labs and exercises with the additional files needed to work.

\subsection{Technologies}

\section{Results}

\section{Future}